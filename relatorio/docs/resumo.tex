%%%%%%%%%%%%%%%%%%%%%%%%%%%%%%%%%%%%%%%%%%%%%%%%%%%%%%%%%%%%%%%%%%%%%%%%%%%%%%%%
% RESUMO %% obrigatório

\begin{resumo}

%% neste arquivo resumo.tex
%% o texto do resumo e as palavras-chave têm que ser em Português para os documentos escritos em Português
%% o texto do resumo e as palavras-chave têm que ser em Inglês para os documentos escritos em Inglês
%% os nomes dos comandos \begin{resumo}, \end{resumo}, \palavraschave e \palavrachave não devem ser alterados

\hypertarget{estilo:resumo}{} %% uso para este Guia

Neste trabalho, dados de fluxo solar na faixa de 10.7 cm são manipulados de modo a simular amostragem não uniforme dos mesmos. O objetivo é investigar o efeito da amostragem não uniforme e introduzir o periodograma de Lomb-Scargle, conceito pertinente à disciplina Análise Wavelet I. Tal ferramenta é discutida e implementada a partir do pacote \texttt{astropy} (da linguagem \texttt{Python}) com a classe \texttt{LombScargle}. Os diferentes periodogramas gerados são analisados e a performance da ferramenta investigada à luz do conhecido ciclo de atividade solar, cujo período é de onze anos. %O objetivo deste projeto é a familiarização de técnicas para análise de séries temporais através do formalismo de Fourier, contribuindo para a assimilação de conceitos relevantes à disciplina Análise Wavelet I.

\palavraschave{%
	\palavrachave{Fluxo solar}%
	\palavrachave{Análise de sinal}%
	\palavrachave{Periodograma de Lomb-Scargle}%
	\palavrachave{Método dos mínimos quadrados}%
	\palavrachave{Séries temporais}%
}
 
\end{resumo}