%%%%%%%%%%%%%%%%%%%%%%%%%%%%%%%%%%%%%%%%%%%%%%%%%%%%%%%%%%%%%%%%%%%%%%%%%%%%%%%
\chapter{CONSIDERAÇÕES FINAIS}

As atividades realizadas no presente trabalho tiveram como objetivo introdução à ferramenta conhecida como periodograma de Lomb-Scargle. Os dados de fluxo solar na faixa de 10.7 cm foram manipulados a fim de simular as condições em que o periodograma de Lomb-Scargle é aplicado: na análise de séries temporais com amostragem aleatória. Os experimentos foram realizados em dois cenários: (1) com $N$ intervalos aleatoriamente distribuídos sobre os dados e de tamanho igual a 10\% do mesmo, e (2) com exclusão aleatória de amostras até $p$\% de amostras restantes. Cinco valores de $N$ e cinco valores de $p$ foram aplicados sobre os dados do fluxo solar, e em seguida seu periodograma produzido com a classe \texttt{LombScargle} do pacote \texttt{astropy} (da linguagem \texttt{Python}).

Os resultados obtidos com a ferramenta empregada foram consistentes na maioria dos testes, indicando robustez da mesma e domínio de seu uso durante a análise. O parâmetro \texttt{nyquist\_factor} empregado foi variado de modo a testar um valor ideal, uma vez que a heurística padrão não permitia enxergar adequadamente a frequência principal na maioria dos testes. Com o valor de 3, o periodograma tinha performance muito ruim na maioria dos testes. A partir deste valor, a ferramenta passou a retornar um pico para a frequência de valor 1, ou seja, um alias persistia nas análises. Para o valor de 2, a ferramenta se tornou consistente e com performance vastamente superior à heurística padrão.

%Com relação a possíveis melhorias na análise, a escolha da melhor frequência foi obtida através da posição do pico nos periodogramas. Esse método pode não ser ideal, pois leakages foram recorrentes durante os testes. Por vezes o pico ocorria em um dos extremos do intervalo de frequência testado pela ferramenta. Um método mais cuidadoso deve ser empregado, ou um teste de diferentes  \texttt{nyquist\_factor} realizados antes da determinação da frequência principal. Além disso, a incerteza das medidas pode ser passada como input à classe \texttt{LombScargle} implementada, o que tornaria a análise ainda mais robusta.

Em resumo, os diferentes efeitos da amostragem foram explorados. Em particular, o efeito de amostragem não uniforme. Sob tal condição, o espectro de potência via FFT não é mais aplicável e o periodograma de Lomb-Scargle se torna a ferramenta ideal. Num pipeline de análise, sua implementação através da classe \texttt{LombScargle} do pacote \texttt{astropy} requer cuidados com a escolha da heurística. Durante os testes aqui realizados, essa ferramenta foi capaz de corretamente identificar o período de $\sim$11 anos do ciclo solar através dos dados de média diária do fluxo F10.7. O ajuste de mínimos quadrados para análise espectral se mostrou robusto na maioria dos cenários testados, e a senóide recuperada se ajustou bem aos dados originais. %Vale ressaltar que a série analisada é não linear, uma vez, durante um ciclo, seu pico cresce mais rapidamente que seu vale decresce.

%Foi tudo ok. Vale ressaltar que esta série é aproximadamente não linear, pois decresce mais rapidamente que cresce. Essa característica não será capturada pelo ajustes de funções senos, exigindo técnicas mais específicas como redes neurais. 

